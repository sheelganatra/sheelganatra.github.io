%!TEX TS-program = xelatex

\documentclass[12pt]{article}
\usepackage {amsmath, amscd, amsbsy, amsfonts, amsthm, eucal}
\usepackage{latexsym,amssymb,mathrsfs,bbm}
\textwidth 6.5in
\topmargin -.3in
\oddsidemargin 0in
\textheight 9in
\parindent 0 pt

\usepackage{color}
\definecolor{rltred}{rgb}{0.75,0,0}
\definecolor{rltgreen}{rgb}{0,0.5,0}
\definecolor{rltblue}{rgb}{0,0,0.75}
\def\red{\color{red}}
\def\black{\color{black}}
\def\green{\color{rltgreen}}
\def\blue{\color{rltblue}}

\newtheorem{lemma}{Lemma}

\newcommand\bi{\begin{itemize}}
\newcommand\ei{\end{itemize}}
\newcommand\beq{\begin{equation}}
\newcommand\eeq{\end{equation}}
\newcommand\itema{\item[(a)]}
\newcommand\itemb{\item[(b)]}
\newcommand\itemc{\item[(c)]}
\newcommand\itemd{\item[(d)]}
\newcommand\iteme{\item[(e)]}
\newcommand\itemf{\item[(f)]}

\renewcommand\and{\qquad\text{and}\qquad}
\renewcommand\|{\ | \ }
\newcommand\ra{\rightarrow}
\newcommand\sr{\stackrel}
\newcommand\mf\mathfrak
\newcommand\mc\mathcal

\def\manylinedefinition#1{
\left\{\begin{array}{ll}
#1
\end{array}\right.
}

\newcommand\N{\mathbb{N}}
\newcommand\Q{\mathbb{Q}}
\newcommand\R{\mathbb{R}}
\newcommand\C{\mathbb{C}}

\def\pb#1{{\green \bf Problem #1.}\hskip 8pt \black}
\def\sol{\textbf{Solution:}}

\newcommand\ev{\text{ev}}
\newcommand\ltwo{\ell^2}
\newcommand\mfa{\mf a}
\def\sequence#1{$\{{#1}_n\}$}
\def\subsequence#1{$\{{#1}_{n_k}\}$}
\def\sumint#1{\sum_{#1=1}^\infty}
\def\sumzero#1{\sum_{#1=0}^\infty}
\def\sumseries#1#2{$\sumint#1 #2_{#1}$}
\newcommand\e\varepsilon
\newcommand\limn{\lim_{n \ra \infty}}
\newcommand\limsupn{\limsup_{n \ra \infty}}
\newcommand\liminfn{\liminf_{n \ra \infty}}
%\newcommand\iff{\quad\Leftrightarrow\quad}

\newcommand\Fixaneps{Fix an arbitrary $\e > 0$ }
\newcommand\foranyeps{for any $\e > 0$ }
\newcommand\givenanyn{given any $N \in \N$ }
\newcommand\thereexistsn{there exists an $n \geq N$ }
\newcommand\thereexistsN{there exists an $N$ such that for every $n \geq N$ }
\newcommand\foreveryn{for every $n \geq N$ }
\newcommand\foreveryk{for every $k \geq N$ }
\newcommand\foreveryx{for every $x \in X$ }
\newcommand\Foreveryx{For every $x \in X$  }

\newcommand\textbook{Johnsonbaugh and Pfaffenberger}
\newcommand\notes{Prof. Simon's notes}

\title{hw4-solutions}

\begin{document}
\centerline{\Large Math 171 Homework 4}
\centerline{\small (due April 29)}
\vskip .2in

%%%%%%%%%%%%%%%%%%%%%%%%%%%%%%%%%%%%%%%%%%%%%%%%%%%%%%%%%%%%%%%%%%%%%%%%%%%%%%%
%problem 38.4
%%%%%%%%%%%%%%%%%%%%%%%%%%%%%%%%%%%%%%%%%%%%%%%%%%%%%%%%%%%%%%%%%%%%%%%%%%%%%%%

\pb{38.4}

Let $M$ be a metric space such that $M$ is a finite set. Prove that every
subset of $M$ is closed.

\sol

Every subset of $M$ is finite, so it is closed by Corollary 38.7.

%%%%%%%%%%%%%%%%%%%%%%%%%%%%%%%%%%%%%%%%%%%%%%%%%%%%%%%%%%%%%%%%%%%%%%%%%%%%%%%
%problem 39.5
%%%%%%%%%%%%%%%%%%%%%%%%%%%%%%%%%%%%%%%%%%%%%%%%%%%%%%%%%%%%%%%%%%%%%%%%%%%%%%%

\pb{39.5}

Prove that the interior of a rectangle in $\R^2$
\[
(a, b) \times (c, d) = \{(x, y) \| a < x < b, c < y < d\}
\]
is an open subset of $\R^2$.

\sol

Let $p_0 = (x_0, y_0)$ be a point in $(a, b) \times (c, d)$ and let
$\e = \min(x_0 - a, b - x_0, y_0 - c, d - y_0)$. We will show that 
$B_{\e}(p_0)$ is contained in $(a, b) \times (c, d)$. Indeed, let
$(x, y)$ be a point in $B_{\e}(p_0)$. Then
\[
|x - x_0|^2 \leq |x - x_0|^2 + |y - y_0|^2 = d_{\R^2}((x, y), (x_0, y_0))
\leq \e^2,
\]
and consequently
\[
|x - x_0| < \e.
\]
Therefore,
\[
a \leq x_0 - \e < x < x_0 + \e \leq b. 
\]
Similarly,
\[
|y - y_0| \leq d_{\R^2}((x, y), (x_0, y_0)) < \e,
\]
so that
\[
c \leq y_0 - \e < y < y_0 + \e \leq d. 
\]
Thus, $(x, y) \in (a, b) \times (c, d)$, as desired.

%%%%%%%%%%%%%%%%%%%%%%%%%%%%%%%%%%%%%%%%%%%%%%%%%%%%%%%%%%%%%%%%%%%%%%%%%%%%%%%
%problem 40.7
%%%%%%%%%%%%%%%%%%%%%%%%%%%%%%%%%%%%%%%%%%%%%%%%%%%%%%%%%%%%%%%%%%%%%%%%%%%%%%%

\pb{40.7}

Let $f$ be a function from a metric space $(M_1, d_1)$ into a metric space
$(M_2, d_2)$. Let $a \in M_1$. Prove that the following are equivalent.

\bi
\itema
$f$ is continuous at $a$.
\itemb
If $U$ is an open subset of $M_2$ which contains $f(a)$, there exists an open
subset $V$ of $M_1$ which contains $a$ such that $V \subset f^{-1}(U)$.
\ei

\sol

\bi
\item (a) $\Rightarrow$ (b) 

Assume $f$ is continuous at $a$. Given an open subset $U$ of $M_2$
containing $f(a)$, by the openness condition there exists an $\e > 0$ such
that $B_{\e}(f(a))$ is contained in $U$. By continuity of $f$, there exists
$\delta > 0$ such that for all $x$ such that $d_1(a, x) < \delta$ we have
that $d_2(f(a), f(x)) < \e$. Then we can take $V := B_{\delta}(a)$ because
$B_{\delta}(a)$ is open (Theorem 39.4), $a \in B_{\delta}(a)$ and
\[
f(B_{\delta}(a)) \subset B_{\e}(f(a)) \subset U,
\]
so that
\[
B_{\delta}(a) \subset f^{-1}(U).
\]

\item (b) $\Rightarrow$ (a) 

Assume that for every open subset $U$ of $M_2$ containing $f(a)$ there
exists an open subset $V$ containing $a$ contained in $f^{-1}(U)$.

Given an arbitrary $\e > 0$, let $U := B_{\e}(f(a))$. By Theorem 39.4
$U$ is open, so there exists an open subset $V$ containing $a$ contained
in $f^{-1}(B_{\e}(f(a)))$. Since $V$ is open, there exists $\delta > 0$ such
that $B_{\delta}(a) \subset V$. Then
\[
B_{\delta}(a) \subset V \subset f^{-1}(B_{\e}(f(a)))
\]
which can be translated into: for all $x \in M_1$ with $d_1(x, a) < \delta$
we have that $d_2(f(x), f(a)) < \e$. Thus, $f$ is continuous at $a$,
as desired.
\ei


%%%%%%%%%%%%%%%%%%%%%%%%%%%%%%%%%%%%%%%%%%%%%%%%%%%%%%%%%%%%%%%%%%%%%%%%%%%%%%%
%problem 40.14a
%%%%%%%%%%%%%%%%%%%%%%%%%%%%%%%%%%%%%%%%%%%%%%%%%%%%%%%%%%%%%%%%%%%%%%%%%%%%%%%

\pb{40.14a}

Let $(M, d)$ be a metric space and let $X$ be a subset of $M$. If $x \in M$,
we define 
\[
d(x, X) := \inf \{d(x, y) \| y \in X\}.
\]
Prove that $f(x) = d(x, X)$ defines a continuous real-valued function
on $M$.

\sol

It suffices to show that whenever $d(x, x')< \e / 2$ we
have that $|d(x, X) - d(x', X)| < \e$.

Given $x \in M$,
choose $y \in X$ such that $d(x, y) < d(x, X) + \e / 2$
(we can do that because by assumption $d(x, X) + \e / 2$ is not
an lower bound of the set $\{d(x, y) \| y \in X\}$).
Then for every $x' \in X$ such that $d(x, x') < \e / 2$ we have that
\[
d(x', X) \leq d(x', x) + d(x, y) < \frac \e 2 + d(x, X) + \frac \e 2 =
d(x, X) + \e.
\]

Similarly, by choosing $y' \in X$ with $d(x', y') < d(x', X) + \e / 2$
we get that
\[
d(x, X) < d(x', X) + \e.
\]

Thus, 
\[
-\e < d(x, X) - d(x', X) < \e,
\]
as desired.

%%%%%%%%%%%%%%%%%%%%%%%%%%%%%%%%%%%%%%%%%%%%%%%%%%%%%%%%%%%%%%%%%%%%%%%%%%%%%%%
%problem 40.17b
%%%%%%%%%%%%%%%%%%%%%%%%%%%%%%%%%%%%%%%%%%%%%%%%%%%%%%%%%%%%%%%%%%%%%%%%%%%%%%%

\pb{40.17b}

Let $M$ be a set and let $d$ and $d'$ be metrics for $M$. We say that $d$
and $d'$ are \emph{equivalent metrics} for $M$ if the collection of open
subsets of $(M, d)$ is identical with the collection of open subsets of
$(M', d')$.

Prove that the metrics $d, d'$, and $d''$ of Exercise 35.7 are equivalent.

\sol

\begin{lemma}
\label{lem:equivalent-metrics}
If every sequence \sequence a in $M$ that converges in $d_1$ also converges
in $d_2$ and vice versa then $(M, d_1)$ is equivalent to $(M, d_2)$.
\end{lemma}

The desired statement follows from the Lemma \ref{lem:equivalent-metrics}
applied to the result of Exercise 37.10 from the previous homework.

\begin{proof}[Proof of Lemma \ref{lem:equivalent-metrics}]
By Theorem 39.5 it suffices to show that a subset $X$ of is closed in 
$(M, d_1)$ if and only if it is closed in $(M, d_2)$. By symmetry
it suffices to only show that if $X$ is closed in $(M, d_1)$, it is
closed in $d_2$.

Assume $X$ is closed in $(M, d_1)$. Let $x$ be a limit point of $X$ in
$(M, d_2)$. Then there exists a sequence \sequence a in $X$ converging
to $x$ in $(M, d_2)$. By the assumption of the problem \sequence a
also converges to $x$ in $(M, d_1)$, so $x$ is a limit point of $X$
in $(M, d_1)$. Since $X$ is closed in $(M, d_1)$, $x \in X$. Since
$x$ was an arbitrary limit point of $X$ in $(M, d_2)$, $X$ is closed
in $(M, d_2)$, as desired.
\end{proof}


%%%%%%%%%%%%%%%%%%%%%%%%%%%%%%%%%%%%%%%%%%%%%%%%%%%%%%%%%%%%%%%%%%%%%%%%%%%%%%%
%problem 1
%%%%%%%%%%%%%%%%%%%%%%%%%%%%%%%%%%%%%%%%%%%%%%%%%%%%%%%%%%%%%%%%%%%%%%%%%%%%%%%

\pb{1. The closure of a set}

Let $(M, d)$ be a metric space and $X \subset M$ a subset.
Recall that $X$ is said to be \emph{closed} if $X = \bar X$, where $\bar X$ 
is the set of limit points of $X$ in $M$.

\bi
\itema
Prove that for any subset $X \subset M$, $\bar X$ is always a closed set. 
(This justifies our use of the terminology \emph{closure} to refer to
$\bar X$.) \\
Note: for this problem you can proceed either directly by definition or
prove that the complement of $\bar X$ is always open, making use of
Theorem 39.5 in the book.
\itemb
Prove that if $X$ is any subset of $M$ with $X \subset Y$ and $Y$ closed,
then $\bar X \subset Y$.
\ei

\sol

\bi
\itema
Let $x$ be a limit point of $\bar X$. Thus, there exists a sequence
$\{x^{(k)}\}_{k \in \N}$ of points in $\bar X$ that converges to $x$. Each
$x^{(k)}$ is a limit point of $X$, so there exists a sequence
$\{x^{(k)}_n\}_{n \in \N}$ converging to $x^{(k)}$. Choose $N_k$ such that
\[
d(x^{(k)}_n, x^{(k)}) < \frac 1 k
\]
for all $n \geq N_k$. 

Consider the sequence $\{x_{N_k}^{(k)}\}_{k \in \N}$. By the triangle
inequality
\[
d(x, x_{N_k}^{(k)}) \leq d(x, x^{(k)}) + d(x^{(k)}_n, x^{(k)}).
\]
Thus, 
\beq
\label{eq:squeeze}
0 \leq d(x, x_{N_k}^{(k)}) < d(x, x^{(k)}) + \frac 1 k.
\eeq
Since, the sequence $\{x^{(k)}\}_{k \in \N}$ converges to $x$, it follows by
Theorem 40.3 and Theorem 40.2 that 
\[\lim_{k \ra \infty} d(x, x^{(k)}) = d(x, x) = 0.
\]
Thus, the right-most term of (\ref{eq:squeeze}) converges to 0. Hence,
by Squeeze Theorem 
\[
\lim_{k \ra \infty} d(x, x_{N_k}^{(k)}) = 0.
\]

By the definition of a limit, for every $\e > 0$ there exists $N$ such that
for every $k \geq N$ we have that $d(x, x_{N_k}^{(k)}) < \e$. Thus, 
the sequence $\{x_{N_k}^{(k)}\}_{k \in \N}$ converges to $x$, so $x$ is 
a limit point of $X$. Thus, $x \in \bar X$, as desired.

\itemb
By definition, $x \in M$ is a limit point of $X$ is there exists a sequence
\sequence x of points in $X$ converging to $x$. Since $X \subset Y$, 
\sequence x is also a sequence of points in $Y$, so $x$ is a limit point of 
$Y$. Since $Y$ is closed, $x \in Y$. Thus, $Y$ contains every limit point of
$X$.
\ei


%%%%%%%%%%%%%%%%%%%%%%%%%%%%%%%%%%%%%%%%%%%%%%%%%%%%%%%%%%%%%%%%%%%%%%%%%%%%%%%
%problem 2
%%%%%%%%%%%%%%%%%%%%%%%%%%%%%%%%%%%%%%%%%%%%%%%%%%%%%%%%%%%%%%%%%%%%%%%%%%%%%%%


\pb{2. The interior of a set}

Let $(M, d)$ be a metric space and $E \subset M$ a subset (not necessarily
open). An \emph{interior point of $E$} is a point $p \in E$ such that 
some $B_{\e}(p) \subset E$. Define the \emph{interior of $E$}, denoted
$\mathring E$, to be the set of all interior points of $E$.

\bi
\itema
Prove that $\mathring E$ is open, and that $E$ is open if and only if
$\mathring E = E$.
\itemb
If $G \subset E$ and $G$ is open, prove that $G \subset \mathring E$.
(In a sense analogous to 1b, this says that $\mathring E$ is the largest
open set contained in $E$).
\itemc
Prove that the complement of the interior $\mathring E^c$ is equal to the
closure of the complement $\overline{E^c}$.
\ei

\sol

\bi
\itema
$\mathring E = E$ \\
$\Leftrightarrow$ Every point $p$ of $E$ is an interior point. \\
$\Leftrightarrow$ For every point $p$ of $E$ there exist $\e > 0$ such
that $B_{\e}(p) \subset E$. \\
$\Leftrightarrow$ E is open.
\itemb
Given any $p$ in $G$, since $G$ is open there exists $\e > 0$ such
that $B_{\e}(p) \subset G$. In particular, $B_{\e}(p) \subset E$.
Hence, $p$ is an interior point of $E$. Thus, $G \subset \mathring E$.
\itemc
Since $\mathring E \subset E$, we have that $\mathring E^c \supset E^c$.
Also, because $\mathring E$ is open, $\mathring E^c$ is closed. Therefore,
by Problem 1b, $\overline{E^c} \subset \mathring E^c$.

We have that $\overline{E^c} \supset E^c$ which
implies $(\overline{E^c})^c \subset E$.
Also, because $\overline{E^c}$ is closed, $(\overline{E^c})^c$ is open.
Therefore, by part b, $(\overline{E^c})^c \subset \mathring E$, which
is equivalent to $\overline{E^c} \supset \mathring E^c$.
\ei


%%%%%%%%%%%%%%%%%%%%%%%%%%%%%%%%%%%%%%%%%%%%%%%%%%%%%%%%%%%%%%%%%%%%%%%%%%%%%%%
%problem 38.4
%%%%%%%%%%%%%%%%%%%%%%%%%%%%%%%%%%%%%%%%%%%%%%%%%%%%%%%%%%%%%%%%%%%%%%%%%%%%%%%

\pb{3. The boundary of a set}
If $X$ is a subset of a metric space, define the boundary of $X$ to be the
set $\partial X : \bar X \cap \overline{X^c}$ (the intersection of the 
closure of $X$ with the closure of the the complement of $X$. Prove that

\bi
\itema
$\partial X$ is closed for any set $X \subset M$.
\itemb
$X \cup \partial X = \bar X$ for any $X$.
\itemc
$X \backslash \partial X = \mathring X$ for any $X$.\\
Note: $\partial X$ is not necessarily strictly contained in $X$. Here
the notation $X \backslash \partial X$ refers to the set of points in
$X$ which are not in $\partial X$.
\ei

\sol

\bi
\itema

By Problem 1a, both $\bar X$ and $\overline{X^c}$ are closed. Hence,
$\partial X$ is closed being the intersection of two closed sets.
\itemb
We have that $X \subset \bar X$ and $\partial X \subset \bar X$
by definition of $\partial X$. Hence $X \cup \partial X \subset \bar X$.
To show that $\bar X \subset X \cup \partial X$, it suffices to prove that
if $x$ is a limit point of $X$ that is not an element of $X$ then 
it is an element of $\partial X$. 

Indeed, by assumption,  and $x \in X^c$, hence $x \in \overline{X^c}$.
Also, by assumption $x \in \bar X$. Hence, 
$x\in \bar X \cap \overline{X^c}$, as desired.
\itemc
\[
X \backslash \partial X = (X^c \cup \partial X)^c 
\]
By part b applied to $X^c$ (and using that $\partial(X^c) = \partial X$ by
definition) the above set equals
\[
(\overline{X^c})^c.
\]
By Problem 2c, it equals
\[
(\mathring X^c)^c,
\]
which in turn equals $\mathring X$, as desired.

\ei


\end{document}