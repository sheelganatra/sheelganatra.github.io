%!TEX TS-program = xelatex

\documentclass[12pt]{article}
\usepackage {amsmath, amscd, amsbsy, amsfonts, amsthm, eucal}
\usepackage{latexsym,amssymb,mathrsfs,bbm}
\textwidth 6.5in
\topmargin -.3in
\oddsidemargin 0in
\textheight 9in
\parindent 0 pt

\usepackage{color}
\definecolor{rltred}{rgb}{0.75,0,0}
\definecolor{rltgreen}{rgb}{0,0.5,0}
\definecolor{rltblue}{rgb}{0,0,0.75}
\def\red{\color{red}}
\def\black{\color{black}}
\def\green{\color{rltgreen}}
\def\blue{\color{rltblue}}

\newtheorem{lemma}{Lemma}
\newtheorem{claim}{Claim}

\newcommand\bi{\begin{itemize}}
\newcommand\ei{\end{itemize}}
\newcommand\beq{\begin{equation}}
\newcommand\eeq{\end{equation}}
\newcommand\itema{\item[(a)]}
\newcommand\itemb{\item[(b)]}
\newcommand\itemc{\item[(c)]}
\newcommand\itemd{\item[(d)]}
\newcommand\iteme{\item[(e)]}
\newcommand\itemf{\item[(f)]}
\newcommand\itemg{\item[(g)]}
\newcommand\itemi{\item[(i)]}
\newcommand\itemii{\item[(ii)]}
\newcommand\itemiii{\item[(iii)]}
\newcommand\itemiv{\item[(iv)]}
\newcommand\itemeq{\item[$\Leftrightarrow$]}

\newcommand\id{\text{id}}

\renewcommand\and{\qquad\text{and}\qquad}
\renewcommand\|{\ | \ }
\newcommand\ra{\rightarrow}
\newcommand\sr{\stackrel}
\newcommand\mf\mathfrak
\newcommand\mc\mathcal

\def\manylinedefinition#1{
\left\{\begin{array}{ll}
#1
\end{array}\right.
}

\newcommand\N{\mathbb{N}}
\newcommand\Q{\mathbb{Q}}
\newcommand\R{\mathbb{R}}
\newcommand\C{\mathbb{C}}

\def\pb#1{{\green \bf Problem #1.}\hskip 8pt \black}
\def\sol{\textbf{Solution:}}

\newcommand\ev{\text{ev}}
\newcommand\ltwo{\ell^2}
\newcommand\mfa{\mf a}
\def\sequence#1{$\{{#1}_n\}$}
\def\subsequence#1{$\{{#1}_{n_k}\}$}
\def\sumint#1{\sum_{#1=1}^\infty}
\def\sumzero#1{\sum_{#1=0}^\infty}
\def\sumseries#1#2{$\sumint#1 #2_{#1}$}
\newcommand\e\varepsilon
\newcommand\limn{\lim_{n \ra \infty}}
\newcommand\limsupn{\limsup_{n \ra \infty}}
\newcommand\liminfn{\liminf_{n \ra \infty}}
%\newcommand\iff{\quad\Leftrightarrow\quad}

\newcommand\Fixaneps{Fix an arbitrary $\e > 0$ }
\newcommand\foranyeps{for any $\e > 0$ }
\newcommand\givenanyn{given any $N \in \N$ }
\newcommand\thereexistsn{there exists an $n \geq N$ }
\newcommand\thereexistsN{there exists an $N$ such that for every $n \geq N$ }
\newcommand\foreveryn{for every $n \geq N$ }
\newcommand\foreveryk{for every $k \geq N$ }
\newcommand\foreveryx{for every $x \in X$ }
\newcommand\Foreveryx{For every $x \in X$  }

\newcommand\textbook{Johnsonbaugh and Pfaffenberger}
\newcommand\notes{Prof. Simon's notes}

\title{hw7-solutions}

\begin{document}
\centerline{\Large Math 171 Homework 7}
\centerline{\small (due May 20)}
\vskip .2in

%%%%%%%%%%%%%%%%%%%%%%%%%%%%%%%%%%%%%%%%%%%%%%%%%%%%%%%%%%%%%%%%%%%%%%%%%%%%%%%
%problem 45.2
%%%%%%%%%%%%%%%%%%%%%%%%%%%%%%%%%%%%%%%%%%%%%%%%%%%%%%%%%%%%%%%%%%%%%%%%%%%%%%%

\pb{45.2}

\bi
\itema
Give an example or a subset of $\R$ which is connected but not compact.
\itemb
Give an example of a subset of $\R$ which is compact but not connected.
\itemc
Characterize the compact connected subsets of $\R$.
\ei

\sol

\bi
\itema
$\R$ is connected by Theorem 45.7 and not compact by Theorem 43.9.
\itemb
$\{0, 1\}$ is not connected because the point $\{0\}$ is open and closed in
$\{0, 1\}$ and compact because it is finite.
\itemc
\begin{claim}
The compact connected subsets of $\R$ are 
\bi
\item the empty set,
\item singleton sets $\{x\}$, $x \in \R$ and
\item closed intervals $[a, b]$, $a, b \in \R$, $a < b$.
\ei
\end{claim}

\begin{proof}
By Corollary 45.4, the connected subsets of $\R$ are the empty set,
the singleton sets, bounded intervals (open, closed and half-open),
rays ($[a, \infty)$ and $(-\infty, a]$) and $\R$ itself. 
By Theorem 43.9 the compact subsets of $\R$ are the closed bounded sets. 
Open intervals are not closed. Rays and $\R$ are not bounded. Hence, 
we get the list in the claim.
\end{proof}
\ei

%%%%%%%%%%%%%%%%%%%%%%%%%%%%%%%%%%%%%%%%%%%%%%%%%%%%%%%%%%%%%%%%%%%%%%%%%%%%%%%
%problem 45.5
%%%%%%%%%%%%%%%%%%%%%%%%%%%%%%%%%%%%%%%%%%%%%%%%%%%%%%%%%%%%%%%%%%%%%%%%%%%%%%%

\pb{45.5}
Let $X$ be a connected subset of a metric space $M$. Prove that $\overline X$ 
is connected. Is $\mathring X$ necessarily connected?

\sol
Assume that $\overline X$ is not connected. We will show that $X$ is also
not connected. Write $\overline X = U \cup V$ where $U$ and $V$ are disjoint
non-empty open subsets of $\overline X$. Then $X = (X \cap U) \cup (X \cap V)$
with $X \cap U$ and $X \cap V$ disjoint open in $X$. To show that $X$
is not connected it suffices to show that $X \cap U$ and $X \cap V$ are 
non-empty. Assume that $X \cap U$ is empty. Then $X = X \cap V$, so 
$X \subset V$.
We show that this implies that $\bar X = V$ contradicting our assumptions
that $U$ in nonempty. Indeed, let $x$ be a limit point of $X$ and let
\sequence x be a sequence of elements of $X$ converging to $x$. Then
each $x_n$ is contained in $V$. Since $V$ is closed in $\bar X$,
and $x$ is a limit point of $V$, $x \in V$. 

Next we give an example when $X$ is connected and $\mathring X$ is not
connected.
Let 
\[
X = ([-2, 0] \times [-2, 0]) \cup ([0, 2] \times [0, 2]).
\]
We show that $\mathring X$ is not connected. Because for every $\e > 0$, 
the point $(-\e/2, \e/2)$ is an element of $B_{\e}((0, 0))$ and 
not of $X$, $(0, 0) \notin \mathring X$. 

Let $f: \R^2 \ra \R$ be given by $f(x, y) = x + y$. 
Since $(0, 0) \notin \mathring X$, $f(x, y) \neq 0$ for every 
$(x, y) \in \mathring X$.


Since 
$f$ is continuous, the set $U := f^{-1}((0, \infty))$ is open
in $\R^2$. Therefore, $U \cap \mathring X$ is open in $\mathring X$.
The set $U \cap \mathring X$ is nonempty because it contains 
the point $(1, 1)$. The set $U \cap \mathring X$ is not all of 
$\mathring X$ because it does not contain the point $(-1, -1) \in \mathring X$.
Finally, $U \cap \mathring X$ is closed in $\mathring X$ because
$U \cap \mathring X = f^{-1}([0, \infty)) \cap \mathring X$.
Thus, $\mathring X$ is not connected.

%%%%%%%%%%%%%%%%%%%%%%%%%%%%%%%%%%%%%%%%%%%%%%%%%%%%%%%%%%%%%%%%%%%%%%%%%%%%%%%
%problem 45.7ab
%%%%%%%%%%%%%%%%%%%%%%%%%%%%%%%%%%%%%%%%%%%%%%%%%%%%%%%%%%%%%%%%%%%%%%%%%%%%%%%

\pb{45.7}

\bi
\itema
Show, by example, that unions and intersections of connected sets are not 
necessarily connected.
\itemb
Prove that if $X$ and $Y$ are connected subsets of $\R$, then
$X \cap Y$ is connected.
\ei

\sol

\bi
\itema
Consider connected singleton sets $\{0\}$ and $\{1\}$. Their union
$\{0, 1\}$ is not connected.


\itemb
By Theorem 45.3, it suffices to show that whenever $a, b \in X \cap Y$ with
$a < b$ then $[a, b] \subset X$. Assume that $a, b \in X \cap Y$. 
Since $a, b \in X$ and $X$ is connected, by Theorem 45.3, 
$[a, b] \subset X$. Similarly, $[a, b] \subset Y$. Thus,
$[a, b] \subset X \cap Y$, as desired.
\ei

%%%%%%%%%%%%%%%%%%%%%%%%%%%%%%%%%%%%%%%%%%%%%%%%%%%%%%%%%%%%%%%%%%%%%%%%%%%%%%%
%problem 46.2
%%%%%%%%%%%%%%%%%%%%%%%%%%%%%%%%%%%%%%%%%%%%%%%%%%%%%%%%%%%%%%%%%%%%%%%%%%%%%%%

\pb{46.2}
Give an example of a complete metric space that is not compact.

\sol

$\R$.

%%%%%%%%%%%%%%%%%%%%%%%%%%%%%%%%%%%%%%%%%%%%%%%%%%%%%%%%%%%%%%%%%%%%%%%%%%%%%%%
%problem 46.3
%%%%%%%%%%%%%%%%%%%%%%%%%%%%%%%%%%%%%%%%%%%%%%%%%%%%%%%%%%%%%%%%%%%%%%%%%%%%%%%

\pb{46.3}
Given an example of a connected metric space that is not complete.

\sol
$\ (0, 1)$ is connected by Corollary 45.4, but not complete because the Cauchy
sequence $\{1/n\}_{n > 1}$ does not converge in $(0, 1)$.

%%%%%%%%%%%%%%%%%%%%%%%%%%%%%%%%%%%%%%%%%%%%%%%%%%%%%%%%%%%%%%%%%%%%%%%%%%%%%%%
%problem 46.5
%%%%%%%%%%%%%%%%%%%%%%%%%%%%%%%%%%%%%%%%%%%%%%%%%%%%%%%%%%%%%%%%%%%%%%%%%%%%%%%

\pb{46.5}
Let $M$ be a metric space.
\bi
\itema
Prove that if $C$ is a complete subset of $M$, then $C$ is closed.
\itemb
Prove that if $M$ is complete, then every closed subset of $M$ is complete.
\ei

\sol
\bi
\itema
Let $x$ be limit point of $C$ in $M$.
Let \sequence x be a sequence of elements of $C$ converging to an element
$x$ of $M$. By Theorem 46.2, \sequence x is a Cauchy sequence in $M$.
Since each $x_n$ is an element of $C$, \sequence x is a Cauchy sequence
in $C$. Since $C$ is complete, \sequence x converges to some $y \in C$
as a sequence in $C$. Thus, \sequence x converges to both $x$ and $y$ as
a sequence in $M$, so $x = y \in C$. Thus, $C$ is contains all of its limit
points.
\itemb
Assume $M$ is complete and let $C$ be a closed subset of $M$. Let
\sequence x be a Cauchy sequence in $C$. Since $M$ is complete, 
\sequence x converges to some $x \in M$. Since $C$ is closed
$x \in C$. Thus, every Cauchy sequence in $C$ converges to an element of $C$.
\ei


%%%%%%%%%%%%%%%%%%%%%%%%%%%%%%%%%%%%%%%%%%%%%%%%%%%%%%%%%%%%%%%%%%%%%%%%%%%%%%%
%problem 1
%%%%%%%%%%%%%%%%%%%%%%%%%%%%%%%%%%%%%%%%%%%%%%%%%%%%%%%%%%%%%%%%%%%%%%%%%%%%%%%

\pb{1}
Let $X$ be any two element set, for instance $\{1, 2\}$, endowed with the
discrete metric. Prove that a metric space $M$ is connected if and only if
continuous functions $f: M \ra X$ are constant.

\sol

Assume that there exists an non-constant function $f: M \ra X$. 
Since the singleton sets $\{1\}$ and $\{2\}$ are open in the discrete metric
and $f$ is continuous it follows that $f^{-1}(\{1\})$ and $f^{-1}(\{2\})$ are
open subsets of $M$. Since $X = \{1\} \cup \{2\}$ it follows that 
\[
M = f^{-1}(\{1\}) \cup f^{-1}(\{2\}).
\]
Since $f$ is non-constant, both $f^{-1}(\{1\})$ and $f^{-1}(\{2\})$ and
are non-empty. Thus, $M$ can be written as a union of two disjoint nonempty 
open sets $f^{-1}(\{1\})$ and $f^{-1}(\{2\})$, so $M$ is not connected.

Assume $M$ is not connected. Say $M = U \cup V$ with $U$ and $V$ disjoint
nonempty open subsets of $M$. Define $f: M \ra X$ by
\[
f(x) \manylinedefinition{ 
	1 & \text{if $x \in U$,} \\
    2 & \text{if $x \in V$.}
}
\]
Then $f$ is continuous because the preimage of every open subset of $X$
is open in $M$. Indeed, $X$ has 4 subsets: $\emptyset$, $\{1\}$, $\{2\}$, $X$,
each of them open in $X$. The preimage of each of the 4 subsets of $X$ 
is open in $M$: $f^{-1}(\emptyset) = \emptyset$, $f^{-1}(\{1\}) = U$,
$f^{-1}(\{2\}) = V$, $f^{-1}(X) = M$.

\end{document}