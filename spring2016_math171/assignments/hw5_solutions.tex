%!TEX TS-program = xelatex

\documentclass[12pt]{article}
\usepackage {amsmath, amscd, amsbsy, amsfonts, amsthm, eucal}
\usepackage{latexsym,amssymb,mathrsfs,bbm}
\textwidth 6.5in
\topmargin -.3in
\oddsidemargin 0in
\textheight 9in
\parindent 0 pt

\usepackage{color}
\definecolor{rltred}{rgb}{0.75,0,0}
\definecolor{rltgreen}{rgb}{0,0.5,0}
\definecolor{rltblue}{rgb}{0,0,0.75}
\def\red{\color{red}}
\def\black{\color{black}}
\def\green{\color{rltgreen}}
\def\blue{\color{rltblue}}

\newtheorem{lemma}{Lemma}

\newcommand\bi{\begin{itemize}}
\newcommand\ei{\end{itemize}}
\newcommand\beq{\begin{equation}}
\newcommand\eeq{\end{equation}}
\newcommand\itema{\item[(a)]}
\newcommand\itemb{\item[(b)]}
\newcommand\itemc{\item[(c)]}
\newcommand\itemd{\item[(d)]}
\newcommand\iteme{\item[(e)]}
\newcommand\itemf{\item[(f)]}
\newcommand\itemi{\item[(i)]}
\newcommand\itemii{\item[(ii)]}
\newcommand\itemiii{\item[(iii)]}

\renewcommand\and{\qquad\text{and}\qquad}
\renewcommand\|{\ | \ }
\newcommand\ra{\rightarrow}
\newcommand\sr{\stackrel}
\newcommand\mf\mathfrak
\newcommand\mc\mathcal

\def\manylinedefinition#1{
\left\{\begin{array}{ll}
#1
\end{array}\right.
}

\newcommand\N{\mathbb{N}}
\newcommand\Q{\mathbb{Q}}
\newcommand\R{\mathbb{R}}
\newcommand\C{\mathbb{C}}

\def\pb#1{{\green \bf Problem #1.}\hskip 8pt \black}
\def\sol{\textbf{Solution:}}

\newcommand\ev{\text{ev}}
\newcommand\ltwo{\ell^2}
\newcommand\mfa{\mf a}
\def\sequence#1{$\{{#1}_n\}$}
\def\subsequence#1{$\{{#1}_{n_k}\}$}
\def\sumint#1{\sum_{#1=1}^\infty}
\def\sumzero#1{\sum_{#1=0}^\infty}
\def\sumseries#1#2{$\sumint#1 #2_{#1}$}
\newcommand\e\varepsilon
\newcommand\limn{\lim_{n \ra \infty}}
\newcommand\limsupn{\limsup_{n \ra \infty}}
\newcommand\liminfn{\liminf_{n \ra \infty}}
%\newcommand\iff{\quad\Leftrightarrow\quad}

\newcommand\Fixaneps{Fix an arbitrary $\e > 0$ }
\newcommand\foranyeps{for any $\e > 0$ }
\newcommand\givenanyn{given any $N \in \N$ }
\newcommand\thereexistsn{there exists an $n \geq N$ }
\newcommand\thereexistsN{there exists an $N$ such that for every $n \geq N$ }
\newcommand\foreveryn{for every $n \geq N$ }
\newcommand\foreveryk{for every $k \geq N$ }
\newcommand\foreveryx{for every $x \in X$ }
\newcommand\Foreveryx{For every $x \in X$  }

\newcommand\textbook{Johnsonbaugh and Pfaffenberger}
\newcommand\notes{Prof. Simon's notes}

\title{hw5-solutions}

\begin{document}
\centerline{\Large Math 171 Homework 5}
\centerline{\small (due May 6)}
\vskip .2in

%%%%%%%%%%%%%%%%%%%%%%%%%%%%%%%%%%%%%%%%%%%%%%%%%%%%%%%%%%%%%%%%%%%%%%%%%%%%%%%
%problem 34.2
%%%%%%%%%%%%%%%%%%%%%%%%%%%%%%%%%%%%%%%%%%%%%%%%%%%%%%%%%%%%%%%%%%%%%%%%%%%%%%%

\pb{34.2}

\bi
\itema
Use Exercise 30.8 and the Heine-Borel theorem to prove that if $f$ is 
continuous on $[a, b]$ and $f(x) > 0$ for every $x$ in $[a, b]$, then there
exists $\e > 0$ such that $f(x) \geq \e$ for every $x \in [a, b]$.
\itemb
Prove the statement in part (a) by considering the function 
$g(x) = 1/f(x)$.
\ei

\sol

\bi
\itema
For every $t \in [a, b]$, consider the function $g_t: [a, b] \ra \R$ given 
by $h_t(x) = f(x) - \frac 1 2f(t)$. We have that $h_t(t) = \frac 1 2 f(t) > 0$
is continuous at $x = t$,
so by Exercise 30.8 there exists $\delta_t > 0$ such that 
$h_t(x) > 0$ for every $x \in (t - \delta_t, t + \delta_t) \cap [a, b]$.

We know that the collection open intervals 
$\{(t - \delta_t, t + \delta_t) \| t \in [a, b]\}$ covers $[a, b]$, so
by Heine-Borel Theorem there exists an finite subcollection
$\{(t_i - \delta_{t_i}, t_i + \delta_{t_i}) \| i = 1, \ldots, n\}$.

Let $\e = \min\{\frac 1 2 f(t_i) \| i = 1, \ldots, n\}$. Then $\e > 0$
and for every $x \in [a, b]$ there exists an $t_i$ such that 
$x \in (t_i - \delta_{t_i}, t_i + \delta_{t_i})$, so 
\[
f(x) - \frac 1 2 f(t_i) = h_{t_i}(x) > 0
\]
and consequently $f(x) > \frac 1 2 f(t_i) \geq \e$.
\itemb
By Theorem 40.4vi, $g$ is continuous on $[a, b]$, so by Theorem 42.6 $g$
is bounded on $M$. Thus, there exists $L > 0$ such that for every 
$x \in [a, b]$ we have that $g(x) \leq M$. Let $\e := M$. Then 
for every $x \in [a, b]$ we have that $f(x) \geq \e$.
\ei


%%%%%%%%%%%%%%%%%%%%%%%%%%%%%%%%%%%%%%%%%%%%%%%%%%%%%%%%%%%%%%%%%%%%%%%%%%%%%%%
%problem 34.6
%%%%%%%%%%%%%%%%%%%%%%%%%%%%%%%%%%%%%%%%%%%%%%%%%%%%%%%%%%%%%%%%%%%%%%%%%%%%%%%

\pb{34.6. (Uniform continuity)}

Prove that if $f$ is continuous on $[a, b]$ and $\e > 0$, then there exists
$\delta > 0$ such that if $|x - y| < \delta$ and $x, y \in [a, b]$, then
$|f(x) - f(y)| < \e$.

\sol

By definition of continuity we know that for every $x \in [a, b]$ there exists
$\delta_x > 0$ such that for every 
$y \in (x - \delta_x, x + \delta_x) \cap [a, b]$
we have that $|f(y) - f(x)| < \e / 2$. Then for any two element $y$ and 
$z$ of $ (x - \delta_x, x + \delta_x) \cap [a, b]$ we have that
\[
|f(y) - f(z)| \leq |f(y) - f(x)| + |f(x) - f(z)| < 
\frac \e 2 + \frac \e 2 = \e.
\]

We know that the collection of open intervals 
$\{(x - \frac {\delta_x} 2, x + \frac {\delta_x} 2) \| x \in [a, b]\}$
covers $[a, b]$. Hence, by Heine-Borel Theorem, there exists finite
subcollection
$\{(x_i - \frac {\delta_{x_i}} 2, x_i + \frac {\delta_{x_i}} 2)
\| i = 1, \ldots, n\}$ that covers $[a, b]$. Let
\[
\delta := \min\left\{ \frac {\delta_{x_i}} 2 \| i = 1, \ldots, n\right\}.
\]

Then given any $x, y \in [a, b]$ with $|x - y| < \delta$, since the
intervals $(x_i - \frac {\delta_{x_i}} 2, x_i + \frac {\delta_{x_i}} 2)$
cover $[a, b]$, there exists $i$ such that 
$x$ is an element of
$(x_i - \frac {\delta_{x_i}} 2, x_i + \frac {\delta_{x_i}} 2)$.

Then
\[
|y - x_i| \leq |y - x| + |x - x_i| < \delta + \frac {\delta_{x_i}} 2 
\leq \delta_{x_i},
\]
so $y$ lies in
$(x_i - \delta_{x_i}, x_i + \delta_{x_i})$. Since $x$ and $y$ are 
both elements of $(x_i - \delta_{x_i}, x_i + \delta_{x_i})$, we have that
$|f(x) - f(y)| < \e$, as desired.
%%%%%%%%%%%%%%%%%%%%%%%%%%%%%%%%%%%%%%%%%%%%%%%%%%%%%%%%%%%%%%%%%%%%%%%%%%%%%%%
%problem 35.9
%%%%%%%%%%%%%%%%%%%%%%%%%%%%%%%%%%%%%%%%%%%%%%%%%%%%%%%%%%%%%%%%%%%%%%%%%%%%%%%

\pb{35.9}

Let $H^\infty$ denote the set of all real sequences \sequence a such that
$|a_n| \leq 1$ for every positive integer $n$. $H^\infty$ is called the
\emph{Hilbert cube}.

\bi
\itema
Let \sequence a, \sequence b$\in H^\infty$. Prove that the series
\[
\sumint n \frac {|a_n - b_n|} {2^n}
\]
converges.
\itemb
Prove that 
\[
d(\{a_n\}, \{b_n\}) := \sumint n \frac {|a_n - b_n|} {2^n}
\]
defines a metric on $H^\infty$.
\ei

\sol

\bi
\itema
We have
\[
|a_n - b_n| \leq |a_n| + |b_n| \leq 2
\]
for every $n$. Hence, the series converges by Comparison test with the
series $\sumint n \frac 1{2^{n - 1}}$.
\itemb
We verify the axioms of a metric:
\bi
\item
Non-negativity:
$d(\{a_n\}, \{b_n\}) \geq 0$, being a sum of a series of non-negative terms.
\item
Distance to self:
$d(\{a_n\}, \{a_n\}) = 0$ because $d(\{a_n\}, \{a_n\})$ is the sum of a 
series consisting of zeros. 
\item
Non-degeneracy: If $d(\{a_n\}, \{b_n\}) = 0$,
then $a_n = b_n$ for every $n$.
\item
Symmetry:
$d(\{a_n\}, \{b_n\}) = d(\{b_n\}, \{a_n\})$ because 
$|a_n - b_n| = |b_n - a_n|$.
\item
Triangle inequality: we know that triangle inequality holds term-wise:
\[
\frac{|a_n - c_n|} {2^n} \leq \frac{|a_n - b_n|}{2^n}
	+ \frac{|b_n - c_n|} {2^n}
\]
for every $n$. Hence, it also holds for partial sums:
\[
\sum_{n=1}^k\frac{|a_n - c_n|} {2^n}
	\leq \sum_{n=1}^k \frac{|a_n - b_n|}{2^n}
    + \sum_{n=1}^k \frac{|b_n - c_n|} {2^n}.
\]
Taking the limit of both sides as $k$ goes to $\infty$ and applying 
Squeeze Theorem we get the desired triangle inequality for $d$:
\[
d(\{a_n\}, \{c_n\}) \leq d(\{a_n\}, \{b_n\}) + d(\{b_n\}, \{c_n\}).
\]
\ei
\ei


%%%%%%%%%%%%%%%%%%%%%%%%%%%%%%%%%%%%%%%%%%%%%%%%%%%%%%%%%%%%%%%%%%%%%%%%%%%%%%%
%problem 41.4
%%%%%%%%%%%%%%%%%%%%%%%%%%%%%%%%%%%%%%%%%%%%%%%%%%%%%%%%%%%%%%%%%%%%%%%%%%%%%%%

\pb{41.4}

Let $M$ be a metric space and let $X$ be a subset of $M$ with the relative
metric. Prove that if $f$ is a continuous function on $M$, then its 
restriction $f|_{X}$ to $X$ is a continuous function on $X$.

\sol

We use the sequential characterization of continuity as in Theorem 40.2.

Let \sequence x be a sequence of elements of $X$ converging to $x$
in the relative metric. Then \sequence x also converges to $x$ viewed 
as a sequence in $M$, hence the sequence $\{f(x_n)\}$ converges to $f(x)$
which is the same as saying that the sequence $\{f|_{X}(x_n)\}$ converges
to $f|_{X}(x)$ because $f|_X(y) = f(y)$ for all $y \in X$.

%%%%%%%%%%%%%%%%%%%%%%%%%%%%%%%%%%%%%%%%%%%%%%%%%%%%%%%%%%%%%%%%%%%%%%%%%%%%%%%
%problem 42.1
%%%%%%%%%%%%%%%%%%%%%%%%%%%%%%%%%%%%%%%%%%%%%%%%%%%%%%%%%%%%%%%%%%%%%%%%%%%%%%%

\pb{42.1}

Prove that none of the spaces $\R^n, \ell^1, \ell^2, c_0, \ell^\infty$ is 
compact.

\sol

By Theorem 42.6 it suffices to produce an unbounded continuous function $f$ on 
each of these spaces to show that they are not compact. We will choose
the function of the form $f(x) := d(x, a)$ where $a$ is a fixed point in
the respective space. Such an $f$ is continuous by Theorem 40.3.

For $\R^n$ pick $a$ to be the zero vector $\underline 0 = (0, \ldots, 0)$.
For every $L > 0$, let $x := (L + 1, 0, 0, \ldots, 0)$. Then
\[
f(x) = d(x, \underline 0) = L + 1 > L.
\]
Thus, $f$ is unbounded.

For all the other spaces we pick $a$ to be the zero sequence $\underline 0$:
$\underline 0_n = 0$. For every $L > 0$, let $x$ be the sequence whose
first element is $L + 1$ and the rest are 0: $x_1 = L + 1$ and $x_n = 0$
for $n > 1$. Note that $x$ is an element of each of the spaces
$\ell^1, \ell^2$, $c_0$ and $\ell^\infty$.

Then 
\[
f(x) = d(x, \underline 0) = L + 1 > L
\]
with respect to each of the metrics $\ell^1, \ell^2$ and $\ell^\infty$.

%%%%%%%%%%%%%%%%%%%%%%%%%%%%%%%%%%%%%%%%%%%%%%%%%%%%%%%%%%%%%%%%%%%%%%%%%%%%%%%
%problem 42.2
%%%%%%%%%%%%%%%%%%%%%%%%%%%%%%%%%%%%%%%%%%%%%%%%%%%%%%%%%%%%%%%%%%%%%%%%%%%%%%%

\pb{42.2}

Let $X$ be a compact subset of a metric space $M$. Prove that $X$ is closed.

\sol

By Theorem 39.5 it suffices to show that the complement $X^c$ of $X$ is
open. Let $y$ be an arbitrary point of $X^c$. Let $f: M \ra \R$ be 
a function given by $f(x) = d(x, y)$. By Theorem 40.3 $f$ is continuous.

Let $U_n := f^{-1}((1/n, \infty))$. By Theorem 40.5iii $U_n$ is open.
Also, for every $x \in X$ we have that $d(x, y) > 0$, so there exists
$n \in \N$ such that $1/ n < d(x, y)$ and consequently
$x \in U_n$. Therefore, $\{U_n\}_{n \in \N}$ forms an open cover of $X$.
Since $X$ was compact, there exists an open subcover $\{U_{n_k}\}$ 
of $X$. Let $n_j$ be the largest among $n_k$'s. Then 
$U_{n_k} \subset U_{n_j}$ for every $k$. Therefore, $X \subset U_{n_j}$, i.e.
for all points $x \in X$ we have $d(x, y) > 1 / n$. 

In particular, for every $z \in M$ such that $d(z, y) < 1 / {n_j}$
we have that $z \in X^c$, so $X^c$ contains an open ball of radius $1/{n_j}$ 
around $y$. Thus, $X^c$ is open.

%%%%%%%%%%%%%%%%%%%%%%%%%%%%%%%%%%%%%%%%%%%%%%%%%%%%%%%%%%%%%%%%%%%%%%%%%%%%%%%
%problem 1
%%%%%%%%%%%%%%%%%%%%%%%%%%%%%%%%%%%%%%%%%%%%%%%%%%%%%%%%%%%%%%%%%%%%%%%%%%%%%%%

\pb{1. Closures and continuity}
Let $M$ and $N$ be metric spaces. Show that the following are equivalent:

\bi
\itemi
$f: M \ra N$ is continuous;
\itemii
$f(\bar A) \subset \overline{f(A)}$ for all sets $A \subset M$;
\itemiii
$\overline{f^{-1}(B)} \subset f^{-1}(\bar B)$ for all subsets $B \subset N$.
\ei

\sol

\bi
\item (ii) $\Rightarrow$ (i)
Assume (ii). By Theorem 40.2 to prove continuity of $f$ it suffices to show
that the sequence $\{f(x_n)\}$ converges to $f(x)$ for every sequence
\sequence x of elements of $M$ converging to $x \in M$. 

We will show the previous statement by contradiction. 
Assume a \sequence x in $M$ converges to $x \in M$ and 
the sequence $\{f(x_n)\}$ \emph{does not} converge to $f(x)$. 
Then, by definition of convergence, there exists $\e > 0$ such that 
for every $N$ there exists $n \geq N$ with $d(f(x_n), f(x)) \geq \e$. 

Hence, there exists a subsequence $\{f(x_{n_k})\}$ of $\{f(x_n)\}$
such that $d(f(x_{n_k}), f(x)) \geq \e$ for every $k$.

Let $A$ be the set of values of the subsequence \subsequence x.
By construction, for every $p \in A$, $d(f(p), f(x)) \geq \e$.
We know that \subsequence x converges to $x$, being a subsequence of 
\sequence x. Therefore, $x \in \bar A$.

By assumption (ii), $f(x) \in \overline{f(A)}$. Therefore, there exists
a sequence \sequence y of elements of $f(A)$ converging to $f(x)$. 
So, there exists $n$ such that $d(y_n, f(x)) < \e$. Since 
$y_n$ is an element of $f(A)$, it can be written as 
$y_n = f(p)$ for some $p \in A$. However, $d(f(p), f(x)) \geq \e$,
so we got the desired contradiction.

\item (iii) $\Rightarrow$ (ii)
Assume (iii). Let $A$ be a subset of $M$ and let $x$ be any element of 
$\bar A$.
Fix a sequence \sequence x of
elements of $A$ converging to $x$. It suffices to show that 
$f(x) \in \overline{f(A)}$.

Let $B = f(A)$. For each $x_n$, $f(x_n)$ is an element of $f(A)$, so
$x_n$ is an element of $f^{-1}(B)$. Therefore, $x \in \overline{f^{-1}(B)}$.
By assumption (iii), $x \in f^{-1}(\bar B)$ which implies $f(x) \in \bar B$,
as desired.

\item (i) $\Rightarrow$ (iii)
Assume (i). Let $B$ be a arbitrary subset of $N$ and let $x$ be an
arbitrary element of $\overline{f^{-1}(B)}$. Fix a sequence \sequence x
of element of $f^{-1}(B)$ converging to $x$. By assumption (i), the sequence
$\{f(x_n)\}$ converges to $f(x)$. For every $n$, since $x_n \in f^{-1}(B)$
we have that $f(x_n) \in B$. Therefore, the limit $f(x)$ of $\{f(x_n)\}$
is a limit point of $B$: $f(x) \in \bar B$. Thus, 
$x \in f^{-1}(\bar B)$, as desired.
\ei

%%%%%%%%%%%%%%%%%%%%%%%%%%%%%%%%%%%%%%%%%%%%%%%%%%%%%%%%%%%%%%%%%%%%%%%%%%%%%%%
%problem 2
%%%%%%%%%%%%%%%%%%%%%%%%%%%%%%%%%%%%%%%%%%%%%%%%%%%%%%%%%%%%%%%%%%%%%%%%%%%%%%%

\pb{2. Closures and interiors in the relative metric}
(Note: this is basically book problem 41.5 rewritten with the notation we've
been using in class and homework.)

\bi
\itema
Let $M$ be a metric space and $X \subset M$ a subset endowed with the relative
metric. If $Y$ is a subset of $X$, let $\bar Y^X$ denote the closure of $Y$ in
the metric space $X$. Prove that $\bar Y^X = \bar Y \cap X$.
\itemb
Recall that we defined the \emph{interior $\mathring B$} of a set 
$B \subset N$ in a
metric space $N$ on last week's homework. If $Y \subset X \subset M$ as above,
state and prove a corresponding result to (a) comparing the interior of $Y$ in
$X$ to the interior of $Y$ in $M$ (also, introduce notation for the two 
different notions.)
\ei

\sol

\bi
\itema
We get the desired result via the following sequence of equivalences. 
\bi
\item[]
$x \in \bar Y^X$.
\item[$\Leftrightarrow$]
$x \in X$ and there exists a sequence \sequence x in $Y \subset X$
converging to $x$
with respect to the relative metric $d_X$ on $X$.
\item[$\Leftrightarrow$]
$x \in X$ and there exists a sequence \sequence x in $Y \subset X$
converging to $X$ with to the metric $d_M$ on $M$.
\item[$\Leftrightarrow$]
$x \in \bar Y \cap X$.
\ei

\itemb
Let $\mathring Y^X$ denote the interior of $Y$ in $X$. We show that
if $Y \subset X \subset M$ then
\[
\mathring Y = \mathring Y^X \cap \mathring X.
\]
by proving inclusions both ways.

$\mathring Y \subset \mathring Y^X \cap \mathring X$. Indeed, given 
$y \in \mathring Y$, there exists an open ball $B^M_{\e}(y)$ in $M$ that
is a subset of $Y$. In particular, $B^M_{\e}(y) \subset X$, so 
$y \in \mathring X$. Also, $B^M_{\e}(y) \cap X = B^X_{\e}(y)
$ is the $\e$-ball in $X$
with the relative metric and it is contained in $Y$. Hence, 
$y \in \mathring Y^X$.

$\mathring Y \supset \mathring Y^X \cap \mathring X$. Given 
$y \in  \mathring Y^X \cap \mathring X$, we can find an open ball
$B_{\e_1}^M(y)$ which is a subset of $X$. Since $y \in \mathring Y^X$, 
there exists an open ball $B_{\e_2}^X(y)$ which is a subset of $Y$.
Let $\e = \min\{e_1, e_2\}$. Then any element of $B_{\e}^M(y)$ is an
element of $X$, hence an element of $B_{\e}^X(y)$ and consequently
and element of $Y$. Thus, $y \in \mathring Y$.
\ei

%%%%%%%%%%%%%%%%%%%%%%%%%%%%%%%%%%%%%%%%%%%%%%%%%%%%%%%%%%%%%%%%%%%%%%%%%%%%%%%
%problem 3
%%%%%%%%%%%%%%%%%%%%%%%%%%%%%%%%%%%%%%%%%%%%%%%%%%%%%%%%%%%%%%%%%%%%%%%%%%%%%%%

\pb{3. An interesting example of closed sets in the relative metric}
Regard $\Q$, the set of all rational numbers, as a metric space with the
metric $d(p, q) = |p - q|$ (this is the \emph{relative metric} for the
inclusion $\Q \subset \R$). Let $E$ be the subset of all $p \in \Q$ such that
$2 < p^2 < 3$. Show that $E$ is closed and bounded in $\Q$ but $E$ is not
compact. Is $E$ open in $\Q$?

\sol

Since $\sqrt 2$ and $\sqrt 3$ are not in $\Q$, $E$ is the intersection of a
closed subset $[-\sqrt 3, -\sqrt 2] \cup [\sqrt 2, \sqrt 3]$ with $\Q$, 
and hence closed by Theorem 41.2ii.

It is bounded because for every $p \in E$ we have $|p| < 2$. 

To show that $E$ is not compact we construct an infinite open cover
that does not have a finite subcover as follows. Let 
$U_n = (-\infty, \sqrt 3 - 1 / n) \cap \Q$. Then each $U_n$ is open in $\Q$
by Theorem 41.2i because it is the intersection of an open set with $\Q$.
Together $\{U_n\}$ covers $E$ because every $x \in E$ satisfies $x < \sqrt 3$,
so there exists $n$ such that $1/n < \sqrt 3 - x$ and hence $x \in U_n$.
However, no finite subcollection $\{U_{n_k}\}$ covers $E$. 
Indeed, given any such
finite subcollection let $n = \max \{n_k\}$. By denseness of rationals
we can find a rational number $q$ in $(\sqrt 3 - 1/{n}, \sqrt 3)$.
By construction, this $q$ is an element of $E$ and not an element of any
of the $U_{n_k}$. Thus, no finite subcollection $\{U_{n_k}\}$
covers $E$, so $E$ is not compact.

$E$ is a also open in $\Q$ because it is the intersection of an open set
$(-\sqrt 3, - \sqrt 2) \cup (\sqrt 2, \sqrt 3)$ with $\Q$.

%%%%%%%%%%%%%%%%%%%%%%%%%%%%%%%%%%%%%%%%%%%%%%%%%%%%%%%%%%%%%%%%%%%%%%%%%%%%%%%
%problem 4
%%%%%%%%%%%%%%%%%%%%%%%%%%%%%%%%%%%%%%%%%%%%%%%%%%%%%%%%%%%%%%%%%%%%%%%%%%%%%%%

\pb{4}
\bi
\itema
Let $M$ be a metric space. A subset $X \subset M$ is said to be
\emph{dense} if $\overline X = M$. Show that if $X \subset M$ is dense, then
for any point $p \in M$ and any $\e > 0$, there exists a point $x \in X$
with $x \in B_{\e}(p)$.
\itemb
A metric space $M$ is said to be \emph{separable} if it contains a countable,
dense set. Show that $\R^k$ is separable.
\emph{
Hint: Let $X$ be the set of points with only rational coordinates.
}
\ei


\sol

\bi
\itema
Let $p$ be an arbitrary point of $M$ and $\e$ an arbitrary positive number. 
We have that $p \in \overline X$, so there exists a sequence \sequence p of 
elements of $X$ converging to $p$.
By the definition of convergence, there exists $N$ such that for every 
$n \geq N$, $d(x_n, p) < \e$. In particular, $x_N \in B_\e(p)$.
\itemb
We know that $X$ as defined in the hint is countable, being a finite product
$\Q^k$ of countable sets $\Q$.

We will show that $X$ is dense in $\R^k$. Let 
$\underline x = (x_1, \ldots, x_k)$ be an arbitrary point of $\R^k$. 
Since $\Q$ is dense in $\R$, for every coordinate $x_i$ we can pick a 
sequence $q_i^{(k)}$ in $\Q$ converging to $x_i$. Then by Theorem 37.2 
we know that the sequence $\{\underline q^{(k)}\}$ converges to 
$\underline x$.

\ei


%%%%%%%%%%%%%%%%%%%%%%%%%%%%%%%%%%%%%%%%%%%%%%%%%%%%%%%%%%%%%%%%%%%%%%%%%%%%%%%
%problem 5
%%%%%%%%%%%%%%%%%%%%%%%%%%%%%%%%%%%%%%%%%%%%%%%%%%%%%%%%%%%%%%%%%%%%%%%%%%%%%%%

\pb{5}

A collection $\mc V := \{ V_\alpha\}_{\alpha \in I}$ of open subsets of a
metric space $M$ is said to be a \emph{base} for $M$ If the following is true:
for every $p \in M$ and every open set $U \subset M$ containing $p$ there
exists $V_{\alpha}$ containing $p$ and that is a subset of $U$. In other 
words, every open set in $M$ is the union of a subcollection of the
$\{V_\alpha\}_{\alpha \in I}$.

Prove that every separable metric space has a countable base. 
\emph{
Hint: take all open balls with rational radius whose center lies in some
countable dense subset of $M$.
}

\sol

Let $M$ be a separable metric space. Consider a countable dense subset
$X$ of $M$. As suggested in the hint consider the collection
$\mathcal V := \{ B_{1/n}(x) \| x \in X,\ n \in \N\}
\}$.
We know that $\mathcal V$ is countable because it is equivalent to cartesian
product $X \times \N$ of two countable sets. We show that $\mc V$ forms a 
base of $M$.

Given $p \in M$ and open set $U$ containing $p$, choose $\e > 0$ such
that $B_{\e}(p)$ is contained in $U$. Choose a positive integer $n$ such that
$1/n < \e/2$. By denseness of $X$, we can choose $x \in X$ that is contained
in $B_{1/n}(p)$. Then $p \in B_{1/n}(x)$. 

If we show that $B_{1/n}(x) \subset B_{\e}(p)$ we are done because
then $B_{1/n}(x)$ is an element of $\mc V$ containing $p$ that is a subset 
of $U$. 

Given any $y \in B_{1/n}(x)$, by the triangle inequality
\[
d(y, p) \leq d(y, x) + d(x, p) < \frac 1 n + \frac 1 n < \e,
\]
so $y \in B_{\e}(p)$, as desired.

%%%%%%%%%%%%%%%%%%%%%%%%%%%%%%%%%%%%%%%%%%%%%%%%%%%%%%%%%%%%%%%%%%%%%%%%%%%%%%%
%problem 6
%%%%%%%%%%%%%%%%%%%%%%%%%%%%%%%%%%%%%%%%%%%%%%%%%%%%%%%%%%%%%%%%%%%%%%%%%%%%%%%

\pb{6}

Prove that every compact metric space $K$ has a countable base and therefore
conclude that $K$ is separable. 
\emph{
Hint: for every $n \in \N$, there are finitely many neighborhoods of $1/n$
which cover $K$.
}

\sol

Fix an arbitrary integer $n$. The collection of open sets
$\{B_{1/n}(x) \| x \in K\}$ covers $K$, so by compactness of $K$ there exists
a finite subcollection $\{B_{1/n}(x_k^{(n)}) \| k = 1, \ldots, m_n\}$ that
covers $K$. 

Consider the union $\mc V$ of the finite covers of the form
$\{B_{1/n}(x_k^{(n)}) \| k = 1, \ldots, m_n\}$ over all $n$.
We claim that $\mc V$ forms a base.

Indeed, given any $p \in K$ and an open set $U$ containing $p$, fix an
$\e > 0$ such that $B_{\e}(p) \subset U$. Next, fix a positive integer $n$
such that $1/n < \e/2$. Since $\{B_{1/n}(x_k^{(n)}) \| k = 1, \ldots, m_n\}$
is a cover of $K$, there exits $k$ such that $B_{1/n}(x_k^{(n)}$ contains $p$.

Then $B_{1/n}(x_k^{(n)} \subset B_{\e}(p)$. Indeed, for every 
$y \in B_{1/n}(x_k^{(n)}$ by the triangle inequality we have that
\[
d(y, p) \leq d(y, x_k^{(n)}) + d(x_k^{(n)}, p) < \frac 1 n + \frac 1 n < \e.
\]

Thus, $\mc V$ indeed forms a base of $K$. We know that $\mc V$ is countable,
being a countable union of finite sets.

Finally, we show that if $K$ has a countable base $\mc V$ then $K$ is 
separable.
For every open set $U$ in $\mc V$ pick an element $x$ of $U$. Let $X$ be the
set of the picked elements. Then $X$ is countable because $\mc V$ is.
We show that $X$ is dense, by showing the complement of its closure
$(\overline X)^c$ is empty.

Indeed, assume that $(\overline X)^c$ is non-empty and pick an element 
$p \in (\overline X)^c$. Then there exists an element $U$ of the base
containing $p$ that is a subset of $(\overline X)^c$. However, by construction
of $X$, $X$ contains at least one element of $U$ leading to a contradiction.

%%%%%%%%%%%%%%%%%%%%%%%%%%%%%%%%%%%%%%%%%%%%%%%%%%%%%%%%%%%%%%%%%%%%%%%%%%%%%%%
%problem 7
%%%%%%%%%%%%%%%%%%%%%%%%%%%%%%%%%%%%%%%%%%%%%%%%%%%%%%%%%%%%%%%%%%%%%%%%%%%%%%%

\pb{7. Constructing open and closed sets.}
By Theorem 40.5, $f$ is continuous if and only if the preimage under $f$ of 
any open set (respectively, closed set) is open (respectively, closed).
This suggests an easy way to give examples of open and closed sets in a metric
space $M$: write down a function $f: M \ra \R$, show that $f$ is continuous
and take the preimage under $f$ of an open or closed set in $\R$. Then 
we can take intersections/unions of such sets to get even more open and closed
sets. Using this method:

\bi
\itema
Show that the generalized ellipsoid 
$
E_{a_1,\ldots, a_{k + 1}} = \{(x_1, \ldots, x_{k + 1}) \|
\sum_{i = 1}^{k + 1} a_i x_i^2 = 1 \}
$
is a closed subset of $\R^{k+1}$ for $a_1, \ldots, a_{k + 1}$ fixed positive
real numbers.
\itemb
Show that 
$
V := \{\underline a = \{ a_n\} \in \ell^\infty \|
a_1^2 + a_2 < 1 \text{ and } a_3 > a_4
\}
$
is an open subset of $\ell^\infty$.
\ei


\sol

\bi
\itema
Let $f: \R^{k + 1} \ra \R$ be given by 
\[
f(x_1, \ldots, x_{k + 1}) = \sum_{i = 1}^{k + 1} a_i x_i^2.
\]
The function $f$ is continuous, being a sum of products of continuous 
coordinate functions $x_i$. Then $E_{a_1,\ldots, a_{k + 1}}$ is closed
being the preimage of the closed set $\{1\}$ under $f$.
\itemb
Let $g, h: \ell^\infty \ra \R$ be given by
\[
g(\underline a) = a_1^2 + a_2
\]
and 
\[
h(\underline a) = a_3 - a_4.
\]
Again, the functions $g$ and $h$ are continuous because they are polynomials
in coordinate functions $a_i$ which are continuous. 
Then $V$ is the intersection of open sets $g^{-1}((-\infty, 1))$ and
$h^{-1}((0, \infty))$ and hence open too.
\ei


\end{document}