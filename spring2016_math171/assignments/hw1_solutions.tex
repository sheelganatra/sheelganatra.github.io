%!TEX TS-program = xelatex

\documentclass[12pt]{article}
\usepackage {amsmath, amscd, amsbsy, amsfonts}
\usepackage{latexsym,amssymb,mathrsfs,bbm}
\textwidth 6.5in
\topmargin -.3in
\oddsidemargin 0in
\textheight 9in
\parindent 0 pt

\usepackage{color}
\definecolor{rltred}{rgb}{0.75,0,0}
\definecolor{rltgreen}{rgb}{0,0.5,0}
\definecolor{rltblue}{rgb}{0,0,0.75}
\def\red{\color{red}}
\def\black{\color{black}}
\def\green{\color{rltgreen}}
\def\blue{\color{rltblue}}

\newcommand\bi{\begin{itemize}}
\newcommand\ei{\end{itemize}}
\newcommand\itema{\item[(a)]}
\newcommand\itemb{\item[(b)]}

\renewcommand\and{\qquad\text{and}\qquad}
\renewcommand\|{\ | \ }
\newcommand\ra{\rightarrow}
\newcommand\mf\mathfrak
\newcommand\mc\mathcal

\newcommand\N{\mathbb{N}}
\newcommand\Q{\mathbb{Q}}
\newcommand\R{\mathbb{R}}
\newcommand\C{\mathbb{C}}

\def\pb#1{{\green \bf Problem #1.}\hskip 8pt \black}
\def\sol{\textbf{Solution:}}

\newcommand\mfa{\mf a}
\def\sequence#1{$\{{#1}_n\}$}
\def\subsequence#1{$\{{#1}_{n_k}\}$}
\newcommand\e\epsilon
\newcommand\limn{\lim_{n \ra \infty}}
\newcommand\limsupn{\limsup_{n \ra \infty}}
\newcommand\liminfn{\liminf_{n \ra \infty}}

\newcommand\Fixaneps{Fix an arbitrary $\epsilon > 0$ }
\newcommand\foranyeps{for any $\epsilon > 0$ }
\newcommand\givenanyn{given any $N \in \N$ }
\newcommand\thereexistsn{there exists an $n \geq N$ }
\newcommand\thereexistsN{there exists an $N$ such that for every $n \geq N$ }
\newcommand\foreveryn{for every $n \geq N$ }
\newcommand\foreveryk{for every $k \geq N$ }
\newcommand\foreveryx{for every $x \in X$ }
\newcommand\Foreveryx{For every $x \in X$  }

\newcommand\textbook{Johnsonbaugh and Pfaffenberger}
\newcommand\notes{Prof. Simon's notes}

\title{hw1-solutions}

\begin{document}
\centerline{\Large Math 171 Homework 1}
\centerline{\small (due April 8)}
\vskip .2in

%%%%%%%%%%%%%%%%%%%%%%%%%%%%%%%%%%%%%%%%%%%%%%%%%%%%%%%%%%%%%%%%%%%%%%%%%%%%%%%
%problem 5.4
%%%%%%%%%%%%%%%%%%%%%%%%%%%%%%%%%%%%%%%%%%%%%%%%%%%%%%%%%%%%%%%%%%%%%%%%%%%%%%%
\pb {5.4}

\newcommand\minusinf{- \inf (-X)}
Show that if $X$ is a nonempty subset of $\R$ which is bounded above then
\[
\sup X = \minusinf.
\]

\sol

We need to show that: (a) $\minusinf$ is an upper bound of $X$ and (b)
that if $b$ is an upper bound of $X$ then $\minusinf \leq b$.

\bi
\item[(a)]
\Foreveryx we have that $-x \in -X$, so
\[
\inf (-X) \leq -x.
\]
Multiplying the equality above by $-1$ (see Theorem 4.2v in \textbook):
\[
\minusinf \geq x.
\]
Since the above inequality holds \foreveryx, $\minusinf$ is an upper bound of
$X$.
\item[(b)]
Assume $b$ is an upper bound of $X$. For every $y \in -X$ we have that 
$-y \in X$, so
\[
-y \leq b.
\]
Multiplying the inequality above by $-1$ we get that
\[
-b \leq y.
\]
Since this holds for every $y \in -X$ we have that $-b$ is a lower bound of 
$-X$. Thus, $-X$ is bounded below and 
\[
-b \leq \inf (-X).
\]
Multiplying the above inequality by $-1$ gives us the desired result:
\[
\minusinf \leq b.
\]
\ei

\vskip 12pt

%%%%%%%%%%%%%%%%%%%%%%%%%%%%%%%%%%%%%%%%%%%%%%%%%%%%%%%%%%%%%%%%%%%%%%%%%%%%%%%
%problem 5.7
%%%%%%%%%%%%%%%%%%%%%%%%%%%%%%%%%%%%%%%%%%%%%%%%%%%%%%%%%%%%%%%%%%%%%%%%%%%%%%%
\pb {5.7}

Let $X$ and $Y$ be subsets of $\R$ with least upper bounds $a$ and $b$,
respectively. Prove that $a + b$ is the least upper bound of the set
\[
X + Y := \{ x + y \| x \in X, y \in Y\}.
\]

\sol

\bi
\item[(a)]
We show that $a + b$ is an upper bound of $X + Y$. Given any element $z$ 
of $X + Y$, by definition of $X + Y$, $z$ can be written as 
\[
z = x + y
\]
for some $x \in X$ and $y \in Y$. Since $a$ and $b$ are upper bounds of
$X$ and $Y$, respectively, we have that
\[
x \leq a \and y \leq b.
\]
Hence $x + y \leq a + b$, which is the same as $z \leq a + b$.
Since $z$ was an arbitrary element of $X + Y$, we have that $a + b$ is 
an upper bound of $X + Y$.
\item[(b)]
Next we show that any upper bound of $X + Y$ is greater or equal to $a + b$
by contradiction. Assume that $c$ is an upperbound of $X + Y$ satisfying
\[
c < a + b.
\]

Let $\epsilon := a + b - c > 0$. Since $a$ is the \emph{least} upper bound
of $X$, $a - \e/2$ (which is smaller than $a$) is \emph{not} an upper
bound of $X$, so there exists $x \in X$ such that
\[
a - \frac \e 2 < x.
\]
Similarly, there exists $y \in Y$ such that
\[
b - \frac \e 2 < y.
\]
Hence, 
\[
\left(a - \frac \e 2\right) + \left( b - \frac \e 2 \right) < x + y.
\]
After simplifying we see that the left hand side of the inequality above
is equal to $c$. Thus, 
\[
c < x + y
\]
contradicting the assumption that $c$ is an upper bound of $X + Y$.
\ei

Thus, $a + b$ is the least upper bound of $X + Y$.

\vskip 12pt

%%%%%%%%%%%%%%%%%%%%%%%%%%%%%%%%%%%%%%%%%%%%%%%%%%%%%%%%%%%%%%%%%%%%%%%%%%%%%%%
%problem 9.10
%%%%%%%%%%%%%%%%%%%%%%%%%%%%%%%%%%%%%%%%%%%%%%%%%%%%%%%%%%%%%%%%%%%%%%%%%%%%%%%
\pb {9.10}

Prove that the plane is \emph{not} the union of a countable family of straight
lines.

\sol

Assume the contrary -- that the plane is the union of a countable family
$\{L_n\}_{n \in \N}$ of straight lines.

Consider a different family of lines - the family of horizontal lines 
$\{P_r\}_{r \in \R}$ -- for each real number $r$
we get a different horizontal line $P_{r} = \{y = r\}$.

The family $\{P_r\}_{r \in \R}$ is uncountable since it is equivalent to $\R$.
Hence, there exists a line $P_r$ in this family which is not one of the 
lines in $\{L_n\}$.

By assumption every point $(x, r)$ in $P_r$ is covered by at least
one of the lines $L_n$. For each $x$ choose such an $n(x)$ such that 
$(x, r) \in L_{n(x)}$. Since two distinct lines intersect in at most one
point all of these $n(x)'s$ are distinct: if $P_r$ intersect $L_n$ 
in $(x, r)$ it can't also intersect $L_n$ at $(x', r)$ with $x' \neq x$.

On one hand the set $\{L_{n(x)} \| x \in \R\}$ is uncountable, being 
equivalent to $\R$ (for each $x$ in $\R$ we get a distinct point 
$(x, r)$ of $P_r$ and in turn a distinct line $L_{n(x)}$ passing through
that point).

On the other hand the set $\{L_{n(x)} \| x \in \R\}$ is countable being 
a subset of a countable set $\{L_n\}_{n \in \N}$.

We get a contradiction. Hence, our initial assumption that the plane can be
covered by countably many lines is wrong.

\vskip 12pt

%%%%%%%%%%%%%%%%%%%%%%%%%%%%%%%%%%%%%%%%%%%%%%%%%%%%%%%%%%%%%%%%%%%%%%%%%%%%%%%
%problem 10.5
%%%%%%%%%%%%%%%%%%%%%%%%%%%%%%%%%%%%%%%%%%%%%%%%%%%%%%%%%%%%%%%%%%%%%%%%%%%%%%%
\pb {10.5}

Prove that 
\[
\limn \frac n {n + 2} = 1.
\]
using the definition of a limit (Definition 10.2).

\sol

We have that 
\[
\frac n {n + 2} - 1 = \frac {n - (n + 2)} {n + 2} = - \frac 2 {n + 2}.
\]

Hence,
\[
\left| \frac n {n + 2} - 1 \right| = \frac 2 {n + 2}
\]
for $n \geq 0$.

\Fixaneps. We are be looking for an $N$ such that \foreveryn we will have
$|\frac n {n + 2} - 1| < \e$. Thus, we are trying to solve
\[
\frac 2 {n + 2} < \e.
\]
The latter inequality is equivalent to
\[
\frac 2 \e < n + 2
\]
(via multiplying both sides by $(n + 2)$ and dividing by $\e$).
Thus, if we choose $N$ to be any integer greater than $2 / \e$ we will have
that
\[
\left| \frac n {n + 2} - 1 \right| < \epsilon,
\]
\foreveryn \ as desired.

\vskip 12pt

%%%%%%%%%%%%%%%%%%%%%%%%%%%%%%%%%%%%%%%%%%%%%%%%%%%%%%%%%%%%%%%%%%%%%%%%%%%%%%%
%problem 13.2
%%%%%%%%%%%%%%%%%%%%%%%%%%%%%%%%%%%%%%%%%%%%%%%%%%%%%%%%%%%%%%%%%%%%%%%%%%%%%%%
\pb {13.2}

Let \sequence a be a sequence with limit 0. Prove that
\[
\limn (-1)^n a_n = 0.
\]

\sol

\Fixaneps. By assumption, \thereexistsN we have that
\[
|a_n - 0| < \epsilon,
\]
which is the same as $|a_n| < \epsilon$.

Therefore, \foreveryn we have that
\[
|(-1)^n a_n - 0| = |(-1)^n a_n| = |a_n| < \epsilon,
\]
as desired.

\vskip 12pt

%%%%%%%%%%%%%%%%%%%%%%%%%%%%%%%%%%%%%%%%%%%%%%%%%%%%%%%%%%%%%%%%%%%%%%%%%%%%%%%
%problem 16.7
%%%%%%%%%%%%%%%%%%%%%%%%%%%%%%%%%%%%%%%%%%%%%%%%%%%%%%%%%%%%%%%%%%%%%%%%%%%%%%%
\pb {16.7}

Prove that every convergent sequence \sequence a has a monotone subsequence.

\sol

Let say $l$ is the limit of \sequence a. Because there are infinitely terms
in a sequence then at least one of the following statements holds true:
\bi
\item
there are infinitely many $n$ for which $a_n = l$;
\item
there are infinitely many $n$ for which $a_n > l$;
\item
there are infinitely many $n$ for which $a_n < l$.
\ei

We show that each of the statements above implies existence of a monotone
subsequence.

\bi
\item
Assume there are in are infinitely many $n$ for which $a_n = l$. Then
we can discard all $n$ for which $a_n \neq l$ and get a subsequence
of \sequence a which is constant, and in particular, both decreasing 
and increasing. Note, that the assumption was necessary in order to get
an infinite subsequence rather than a finite set of terms.
\item
Assume there are infinitely many $n$ for which $a_n > l$. Similarly to the
previous part, start by throwing away all $n$ for which $a_n \leq l$. Then
we are left with a subsequence of \sequence a converging to $l$ all of whose
terms are greater than $n$. Let's call this subsequence \sequence b in
order to avoid nested subscripts.

We then recursively choose a decreasing subsequence
\subsequence b of \sequence b (which is going to serve a the desired
monotone subsequence of \sequence a).

Assume we already chose $\{n_k\}$ for all $k \leq m$ such that
$b_{n_k} < b_{n_{k - 1}}$. Then all we 
need to find for the recursive step is $n_{m + 1}$ such that
\bi
\item
$n_{m + 1} > n_m$
\item
$b_{n_{m + 1}} < b_{n_m}$
\ei

Since $b_{n_m} > l$ and \sequence b converges to $l$ we can use 
$b_{n_m} - l$ as $\e$ in the definition of the limit to get that
there exists $N$ such that for all $n \geq N$ we have that
\[
|b_n - l| < b_{n_m} - l
\]
Since by construction of \sequence b, $b_n > l$ the above inequality becomes
\[
b_n - l < b_{n_m} - 1
\]
i.e. $b_n < b_{n_m}$ for all $n \geq N$. Thus, we may take 
$n_{m + 1}$ to be $N$.
\item
Assume there are infinitely many $n$ for which $a_n < l$. In this case
we construct an increasing subsequence following all of the steps of the
previous case.
\ei

\vskip 12pt

%%%%%%%%%%%%%%%%%%%%%%%%%%%%%%%%%%%%%%%%%%%%%%%%%%%%%%%%%%%%%%%%%%%%%%%%%%%%%%%
%problem 16.10
%%%%%%%%%%%%%%%%%%%%%%%%%%%%%%%%%%%%%%%%%%%%%%%%%%%%%%%%%%%%%%%%%%%%%%%%%%%%%%%
\pb {16.10}

\bi
\itema
Let $x$ and $y$ be positive numbers. Let $a_0 := y$, and let
\[
a_n := \frac {(x / a_{n - 1}) + a_{n - 1}} 2 
	\qquad \text{for } n = 1, 2, \ldots
\]
Prove that \sequence a is a decreasing sequence with limit $\sqrt x$.
\itemb
Generalize (a) to $k^{th}$ roots.
\ei

\sol
\bi
\itema
We start by proving $a_n > 0$ by induction on $n$: 
\bi
\item the base case $n = 0$: $a_0 = y > 0$ by assumption;
\item induction step, if $a_{n-1} > 0$ then $x / a_{n-1} > 0$ so
$a_n > 0$.
\ei

Next we show that \sequence a is decreasing for $n \geq 1$. 
We have the following chain of equivalences of inequalities
\begin{align*}
a_n \leq a_{n-1} 
& \Leftrightarrow
\frac {(x / a_{n - 1}) + a_{n - 1}} 2  \leq a_{n - 1} \\
& \Leftrightarrow
\frac x {a_{n - 1}} \leq a_{n-1} \\
& \Leftrightarrow
x \leq a_{n-1}^2 \\
& \Leftrightarrow
\sqrt x \leq a_{n-1}.
\end{align*}

Thus, if we prove that $a_n \geq \sqrt x$ for $n \geq 1$
we get that \sequence a is decreasing for $n \geq 1$.

Note that $a_n$ is the arithmetic mean of two terms 
($x/a_{n-1}$ and $a_{n-1}$) whose geometric means is $\sqrt x$:
\[
\sqrt{x / a_{n-1} \cdot a_{n-1}} = \sqrt x.
\]

Thus, by the AM-GM inequality $a_n \geq \sqrt x$.

Hence, \sequence a is decreasing for $n \geq 1$.

Since, \sequence a is decreasing and bounded below (by $\sqrt x$ as we 
have shown above), it converges (Theorem 16.2).
Let's call its limit $l$.

By Lemma 14.2 applied to the constant sequence $b_n := \sqrt x$
and \sequence a, we have that
\[
\sqrt x \leq l.
\]

We have that 
\[
\limn a_{n-1} = l
\] 
(because $\{a_{n-1}\}$ is that same
sequence as \sequence a up to index shift) and 
\[
\limn \frac x {a_{n - 1}} = \frac x l
\]
by Lemma 12.8 and Theorem 12.3. Therefore, on one hand
\[
\limn \frac {(x / a_{n - 1}) + a_{n - 1}} 2 = \frac {x/l + l} 2
\]
and on the other hand
\[
\limn \frac {(x / a_{n - 1}) + a_{n - 1}} 2 = \limn a_n = l.
\]

We can now solve for $l$:
\[
\frac {x/l + l} 2 = l
\quad\Leftrightarrow\quad
x / l = l
\quad\Leftrightarrow\quad
x = l^2
\quad\Leftrightarrow\quad
l = \sqrt x
\]

Thus, $\limn a_n = \sqrt x$ as desired.

\itemb
The intuition look for a recursive definition of \sequence a where
$a_n$ is an arithmetic mean of terms whose geometric mean equals
$\sqrt[k] x$. It turns out the following statement is true:

Let $x$ and $y$ be positive numbers. Let $a_0 := y$, and let
\[
a_k := \frac {(x/a_{n - 1}^{k - 1}) + (k - 1) a_{n - 1}} k
\]
\ei

Then \sequence a is decreasing for $n \geq 1$ and converges to 
$\sqrt[k] x$.

We prove the more general statement following the same steps as in
part (a):
\bi
\item
We prove that $a_n$ is positive by induction on $n$,
\item
By AM-GM inequality,
\[
a_n \geq \sqrt[k] x
\]
\item
We show that $a_n \leq a_{n - 1}$ is equivalent to 
$a_{n - 1} \geq \sqrt[k] x$ and conclude that \sequence a is decreasing,
\item
Because \sequence a is bounded and decreasing it converges to some $l$
which satisfies 
\[
l = \frac{x / l + (k - 1) l} k.
\]
\item
Solving the latter equation for $l$ we get that $l = \sqrt[k] x$.
\ei

\vskip 12pt

%%%%%%%%%%%%%%%%%%%%%%%%%%%%%%%%%%%%%%%%%%%%%%%%%%%%%%%%%%%%%%%%%%%%%%%%%%%%%%%
%problem 18.5
%%%%%%%%%%%%%%%%%%%%%%%%%%%%%%%%%%%%%%%%%%%%%%%%%%%%%%%%%%%%%%%%%%%%%%%%%%%%%%%
\pb {18.5}
Let $\{a_n\}$ be a sequence such that for some $\epsilon > 0$, $|a_n - a_m| \geq \epsilon$ for all $n\neq m$. Prove that $\{a_n\}$ has no convergent subsequence.

\sol

Let $\{a_n\}$ be as above and suppose that there is a convergent subsequence $\{a_{n_k}\}$ converging to $L$. This means that, for $\epsilon' := \epsilon / 2$, there is an $N > 0$ so that for all $k \geq N$, $|a_{n_k}-L| < \epsilon':= \epsilon/2$. Pick two natural numbers $k\neq k'$ with $k,k'\geq N$. It follows that $|a_{n_k}-L| < \epsilon'$ and $|a_{n_{k'}}-L| < \epsilon'$. Thus, by the triangle inequality, $|a_{n_k} - a_{n_{k'}}| = |a_{n_k} - L + L - a_{n_{k'}}| \leq |a_{n_k}-L| + |a_{n_{k'}}-L| < 2 \epsilon' = \epsilon$. But this is a contradiction to the hypotheses on $\{a_n\}$. We conclude that $\{a_n\}$ does not have a convergent subsequence.

\vskip 12pt

%%%%%%%%%%%%%%%%%%%%%%%%%%%%%%%%%%%%%%%%%%%%%%%%%%%%%%%%%%%%%%%%%%%%%%%%%%%%%%%
%problem 20.13
%%%%%%%%%%%%%%%%%%%%%%%%%%%%%%%%%%%%%%%%%%%%%%%%%%%%%%%%%%%%%%%%%%%%%%%%%%%%%%%
\pb {20.13}

Let \sequence a and \sequence b be sequences such that \sequence a is 
convergent and \sequence b is bounded. Prove that
\[
\limsupn (a_n + b_n) = \limsupn a_n + \limsupn b_n
\]
and
\[
\liminfn (a_n + b_n) = \liminfn a_n + \liminfn b_n.
\]

\sol

Let $c_n := a_n + b_n$ and let $l$ be the limit of $a$. 
Let $\mc L_b$ and $\mc L_c$ are the limit sets
(i.e. sets of limits of all converging subsequences)
of \sequence b and \sequence c, respectively. We will show that
\[
\mc L_c = \mc L_b + \{ l\},
\]
i.e.
\[
\mc L_c = \{ x + l \| x \in L_b \}.
\]

Then it will follow by Problem 5.7 that 
\[
\sup \mc L_c = \sup \mc L_b + l
\]
which (after taking into account that $\limsupn a_n = \limn a_n = l$
by Theorem 20.4) is equivalent to:
\[
\limsupn (a_n + b_n) = \limsupn a_n + \limsupn b_n.
\]

Showing that $\mc L_c = \mc L_b + \{ l\}$ is equivalent to showing that
any real number $x$ is in $\mc L_b$ if and only if $x + l$ is in $\mc L_c$.

\bi
\item
Assume $x \in \mc L_b$. By definition of $\mc L_b$ there exists a
subsequence \subsequence b of \sequence b that converges to $x$. 
By Theorem 11.2 the corresponding subsequence \subsequence a of 
\sequence a also converges to $l$. Hence, by Theorem 12.2
the corresponding subsequence $c_{n_k} = a_{n_k} + b_{n_k}$
converges to $x + l$, so $x + l \in \mc L_c$.
\item
Conversely, assume $x + l \in \mc L_c$. Then there exists a 
subsequence \subsequence c of \sequence c that converges to $x + l$.
Then by the same argument as in the previous part, 
$b_{n_k} = c_{n_k} - a_{n_k}$ converges to $x = (x + l) - l$, so
$x \in \mc L_b$.
\ei

The statement about $\liminf$ follows from $\mc L_c = \mc L_b + \{ l\}$
and the results of Problem 5.4 and Problem 5.7.
\vskip 12pt

%%%%%%%%%%%%%%%%%%%%%%%%%%%%%%%%%%%%%%%%%%%%%%%%%%%%%%%%%%%%%%%%%%%%%%%%%%%%%%%
%problem 1
%%%%%%%%%%%%%%%%%%%%%%%%%%%%%%%%%%%%%%%%%%%%%%%%%%%%%%%%%%%%%%%%%%%%%%%%%%%%%%%
\pb 1

Let \sequence a be a sequence of real numbers. We say a real number $l$ is a 
\emph{cluster point} of \sequence a if \foranyeps, there are infinitely many
$a_n$ within $\e$ of $l$. Formally, \foranyeps, \givenanyn, \thereexistsn
with $|a_n - l| < \e$.

\bi
\item[(a)]
Show that if $l$ is a cluster point of \sequence a, then there exists a 
subsequence of \sequence a converging to $l$.
\item[(b)]
Suppose $\mfa :=$\sequence a is a bounded sequence and let $C_\mfa$ denote
the set of all cluster points of $\mfa$. Show that $C_\mfa$ is bounded
(so its sup and inf exist), and moreover that
\[
\limsup_n\{a_n\} = \sup C_\mfa
\]
\[
\liminf_n\{a_n\} = \inf C_\mfa
\]
\ei

\sol

\bi
\itema
Assume $l$ is a cluster point of \sequence a. We will recursively construct
a subsequence \subsequence a of \sequence a that converges to $l$.
More precisely, we recursively construct a strictly increasing sequence $n_k$
of positive integers such that
\begin{equation}
\label{eq:subseq}
|a_{n_k} - l| < \frac 1 k.
\end{equation}
Assume we have constructed the first $m$ terms of $\{n_k\}$. Then
we can choose the $(m+1)^{st}$ term $n_{m+1}$ satisfying
\[
n_{m+1} >= n_m + 1 \and |a_{n_{m+1}} - l| < \frac 1 {m+1}
\]
using the definition of a cluster point.

Thus, we have constructed a subsequence \subsequence a of \sequence a
satisfying (\ref{eq:subseq}). We show that 
\[
\lim_{k \ra \infty} a_{n_k} = l.
\]

Given $\epsilon > 0$, choose an integer $N > 1 / \epsilon$. Then
\foreveryk we have that 
\[
|a_{n_k} - l| < \frac 1 k < \frac 1 N < \epsilon.
\]
Hence, $\lim_{k \ra \infty} a_{n_k} = l$, as desired.
\itemb
Let $\mc L_\mfa$ be the set of all limits of converging subsequences
of \sequence a. Then we have that
\[
\limsup_n \{a_n\} = \sup \mc L_\mfa \and
\liminf_n \{a_n\} = \inf \mc L_\mfa
\]
(see Definition 20.1). Thus, if we are able to prove that 
\[
\mc L_\mfa = C_\mfa
\]
then we are done.

By part (a) we have that $C_\mfa \subset \mc L_\mfa$, so we are left to prove 
that $\mc L_\mfa \subset C_\mfa$.

Let $l$ be an arbitrary element of $\mc L_\mfa$. We will prove that $l$ 
is an element of $C_\mfa$. By definition of 
$\mc L_\mfa$ there exists a subsequence \subsequence a of \sequence a
converging to $L$. Thus, \foranyeps we can choose $M$ such that 
for every $k \geq M$ we have that 
\begin{equation}
\label{eq:subseq-conv}
|a_{n_k} - l| < \e.
\end{equation}
Given any $N$ we can produce a $k$ with $k \geq M$ and
$n_k \geq N$ (the latter due to $\lim_{k\ra \infty} n_k = \infty$). Thus,
this $k$ will also satisfy (\ref{eq:subseq-conv}). Thus, given an arbitrary
$\epsilon$ and $N$ we have produced $n$ (namely $n := n_k$) such that 
$|a_n - l| < \e$ and $n \geq N$.
\ei

%%%%%%%%%%%%%%%%%%%%%%%%%%%%%%%%%%%%%%%%%%%%%%%%%%%%%%%%%%%%%%%%%%%%%%%%%%%%%%%
%problem 2
%%%%%%%%%%%%%%%%%%%%%%%%%%%%%%%%%%%%%%%%%%%%%%%%%%%%%%%%%%%%%%%%%%%%%%%%%%%%%%%
\pb 2

It may be used without proof that the complex numbers $\C$ satisfy the axioms
of a field (axioms 1-11) in \textbook, or F1-F6 in \notes. Prove that they 
\emph{don't} satisfy the axioms of an
ordered field (axiom 12 in JP or O1-O2 in \notes).

\sol

We'll proceed by contradiction.
Assume that $\C$ did satisfy the axioms of an ordered field, i.e. that there
exists a subset $P$ of $\C$ such that 
\bi
\itema
if $x, y\in P$ then $x + y, xy \in P$ and
\itemb
for every $x\in \C$ either $x=0$, $x \in P$ or $x \in -P$ and no two of the
conditions hold simultaneously.
\ei

By condition (b) exactly one of $i$ and $-i$ is in $P$. Let's call this
number $\alpha$. Then $\alpha \in P$ and $\alpha^2 = -1$. 

By condition (a) with $x = y = \alpha$ we have that $\alpha^2 = -1 \in P$.
Then by condition (a) with $x = \alpha$ and $y = -1$ we have that 
$- \alpha \in P$. Thus, both $\alpha$ and $- \alpha$ are in $P$, thus
contradicting condition (b).

Hence, $\C$ can't satisfy the ordered field axioms.
\end{document}
